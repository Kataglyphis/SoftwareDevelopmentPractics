\subsection{InGame - Interface}

Das Interface ist ähnlich wie ECS aufgebaut, wobei DynamicGameObjects jeweils Entities, IProperties jeweils Components und IPropertyProcessors jeweils Systeme sind.
Der Hauptunterschied zu herkömmlichen ECS ist jedoch, dass mehrere Instanzen einer IProperty auf dem selben DynamicGameObject aktiv sein können
(zum Beispiel eine AnimatorProperty für jede aktive Animation) und jedes DynamicGameObject innerhalb des zuständingen DynamicGameObjectProcessor
seine eigenen Instanzen der IPropertyProcessors hat, da ein IPropertyProcessor nur auf einer einzigen Node arbeiten kann. \par
Um Änderungen am Aufbau der DynamicGameObjects zu kommunizieren gibt es DynamicGameObjectEvents, welche erstellt werden, falls eine IProperty hinzugefügt oder entfernt wird.
Diese DynamicGameObjectEvents werden anhand ihrer Aktion mit verschiedenen Strategien verarbeitet.\par
Analog zu den DynamicGameObjectEvents gibt es TickEvents, welche eine Änderung im Aufbau eines Ticks kommunizieren und ebenfalls anhand einer Aktion auf
verschiedene Strategien aufgeteilt werden.\par