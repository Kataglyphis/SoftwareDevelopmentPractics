\subsection{Assets}

Das Assets Modul verwendet .json Dateien um Modelle und Metadaten zu laden.
Bei den Dateien wird unterschieden zwischen dem AssetPack und den AssetInfos.\par

\subsubsection{AssetPack}

Das AssetPack ist eine Kollektion von benannten Gruppen, welche verweise auf \textit{AssetInfos} speichern.
Die Pfade der \textit{AssetInfos} werden dabei relativ zum AssetPack angegeben. \par
\begin{figure}[htbp]
    \centering
    \lstinputlisting[language=json]{Assets/SamplePack.json}
    \caption{AssetPack Sample}
\end{figure}
\pagebreak

\subsubsection{AssetInfo}

Eine \textit{AssetInfo} beinhaltet einen Verweis auf eine .j3o Datei, welche das Model enthält,
einen Schlüssel, unter dem man es später beim JsonModelProvider wieder finden kann,
eine Kollektion von Nodes im Model die während dem Spiel erreichbar sein müssen, als \textit{NodeInfo},
und eine Kollektion von AbstractControls, welche während des Spiels benötigt werden, als ControlInfo. \par
Eine \textit{NodeInfo} hat dabei einen Schlüssel, mit welchem man sie später über das AssetControl erreichen kann,
und einen Pfad, welcher den Weg von der Wurzel des Models zur Node beschreibt. Zudem enthalten \textit{ControlInfos}
noch einen String, welcher den Typ des AbstractControl festlegt, und einen Boolean, welcher festlegt ob nach dem
Laden des Models durch den AssetManager das AbstractControl noch erstellt werden muss oder nicht. \par
\begin{figure}[htbp]
    \centering
    \lstinputlisting[language=json]{Assets/asset0.json}
    \caption{AssetInfo Sample}
\end{figure}
\pagebreak