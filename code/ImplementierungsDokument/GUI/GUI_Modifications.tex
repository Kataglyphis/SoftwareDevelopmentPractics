\subsection{GUI}

\subsubsection{IConfig}
\begin{itemize}
    \item \textit{Umbenannt zu ConfigManager und kein Interface mehr}
        \begin{leftbar}[0.9\linewidth]
            Hier lag ein Entwurfsfehler vor. Diese Klasse stellt nun Funktionen zum Lesen und 
            Schreiben einer Datei bereit. Sie enthält für jede Methode einen Pfad, wo sich 
            die Datei befinden sollte.
        \end{leftbar}
\end{itemize}

\subsubsection{InternalGameConfig, SeedConfig und GraphicsConfig}
\begin{itemize}
    \item \textit{Implementiert nicht mehr IConfig}
        \begin{leftbar}[0.9\linewidth]
            Da die Klasse statische Methoden enthält, implementiert sie nicht mehr das 
            Interface IConfig.
        \end{leftbar}
    \item \textit{Methode getConfig() hinzugefügt}
        \begin{leftbar}[0.9\linewidth]
            Gibt die Konfiguration zurück und ruft bei die getConfig()
            Methode vom ConfigManager auf.
        \end{leftbar}
    \item \textit{Methode saveConfig() hinzugefügt}
        \begin{leftbar}[0.9\linewidth]
            Speichert die Konfiguration und ruft bei die 
            saveConfig() Methode vom ConfigManager auf.
        \end{leftbar}
    \item \textit{Methode reset() hinzugefügt}
        \begin{leftbar}[0.9\linewidth]
            Setzt die Config zurück durch Überschreiben mit der Default-Datei.
        \end{leftbar}
\end{itemize}

\subsubsection{OwnScreenController}
\begin{itemize}
    \item \textit{Klasse hinzugefügt}
        \begin{leftbar}[0.9\linewidth]
            OwnScreenController implementiert von ScreenController und wird von jedem Menü-
            Bildschirm geerbt. Es enthält Funktionen und Attribute, die jeder Bildschirm 
            braucht und reduziert somit Code-Duplizierung. Jedem OwnScreenController wird 
            außerdem eine Referenz auf den NiftyAppState und die anderen OwnScreenController
            mitgegeben, so dass diese leicht aufrufbar sind.
        \end{leftbar}
\end{itemize}

\subsubsection{NiftyAppState}
\begin{itemize}
    \item \textit{Gesamtes Entwurfskonzept der Klasse geändert}
        \begin{leftbar}[0.9\linewidth]
            Ein NiftyAppState ist nun eine AppState, welcher alle die ihm mitgegebenen 
            OwnScreenController in einer Nifty Instanz initialisiert und speichert.
            Die OwnScreenController innerhalb eines NiftyAppStates haben Zugriff auf 
            einander und können so leicht mit nifty.goToScreen() aufgerufen werden.
            Ist dies nicht möglich, so erstellt ein OwnScreenController ein neues 
            NiftyAppState-Objekt mit den jeweiligen OwnScreenControllern und fügt dieses
            an den AppStateManager an.
            Wird der NiftyAppState vom AppStateManager entfernt, so entfernt dieser alle 
            initialisierten Sachen wie z.B den Processor für das NiftyDisplay.
        \end{leftbar}
\end{itemize}

\subsubsection{SplashScreen}
\begin{itemize}
    \item \textit{Klasse wurde hinzugefügt}
        \begin{leftbar}[0.9\linewidth]
            Zeigt einen SplashScreen bei Start des Programms an.
        \end{leftbar}
\end{itemize}

\subsubsection{MainmenuScreen}
\begin{itemize}
    \item \textit{Implementiert nicht mehr ScreenController, erbt dafür von OwnScreenController}
        \begin{leftbar}[0.9\linewidth]
            Es wurde eine eigene ScreenController Klasse hinzugefügt.
        \end{leftbar}
    \item \textit{Methode openMultiplayerMode() entfernt}
        \begin{leftbar}[0.9\linewidth]
            Wurde nicht mehr benötigt, da die Mehrspieler-Funktion inklusive des 
            Lobby Bildschirmes entfernt wurde.
        \end{leftbar}
\end{itemize}

\subsubsection{AbstractEditGameSettings}
\begin{itemize}
    \item \textit{Klasse entfernt}
        \begin{leftbar}[0.9\linewidth]
            Diese Klasse existierte ursprünglich, da der Lobby Bildschirm und der 
            Einzelspielerbildschirm viele gleiche Methoden brauchten. Da der Lobby 
            Bildschirm entfernt wurde, wurden alle Funktionen hierraus in die Klasse 
            SinglePlayerModeScreen verschoben.
        \end{leftbar}
\end{itemize}

\subsubsection{LobbyScreen}
\begin{itemize}
    \item \textit{Klasse entfernt}
        \begin{leftbar}[0.9\linewidth]
            Wurde nicht mehr benötigt, da die Mehrspieler-Funktion inklusive des 
            Lobby Bildschirmes entfernt wurde.
        \end{leftbar}
\end{itemize}

\subsubsection{SinglePlayerModeScreen}
\begin{itemize}
    \item \textit{Implementiert nicht mehr ScreenController, erbt dafür von OwnScreenController}
        \begin{leftbar}[0.9\linewidth]
            Es wurde eine eigene ScreenController Klasse hinzugefügt.
        \end{leftbar}
    \pagebreak
    \item \textit{Methoden aus AbstractEditGameSettings hier eingefügt}
        \begin{leftbar}[0.9\linewidth]
            AbstractEditGameSettings wurde entfernt und dementsprechend sind dessen Methoden 
            hier eingesetzt worden.
        \end{leftbar}
    \item \textit{Methode editCurrentSeed(String, DropDownSelectionChangedEvent) hinzugefügt}
        \begin{leftbar}[0.9\linewidth]
            Wird aufgerufen sobald eine Auswahl im DropDown-Menü getroffen wird.
        \end{leftbar}
    \item \textit{Methode editCurrentSeedNumber(String, TextFieldChangedEvent) hinzugefügt}
        \begin{leftbar}[0.9\linewidth]
            Wird aufgerufen sobald die Seed Nummer bearbeitet wird.
        \end{leftbar}
    \item \textit{Methode editKICount(String, TextFieldChangedEvent) hinzugefügt}
        \begin{leftbar}[0.9\linewidth]
            Wird aufgerufen sobald die KI Anzahl bearbeitet wird.
        \end{leftbar}
    \item \textit{Methode saveSeedName() hinzugefügt}
        \begin{leftbar}[0.9\linewidth]
            Wird aufgerufen sobald der Seed Name mit Enter bestätigt, oder der Speicher-Button
            rechts daneben geklickt wird.
        \end{leftbar}
    \item \textit{Methode editSeednameTextfieldVisibility() hinzugefügt}
        \begin{leftbar}[0.9\linewidth]
            Wird aufgerufen sobald der Bearbeiten-Button neben dem DropDown-Menü gedrückt wird.
        \end{leftbar}
    \item \textit{Methode saveSeedNumber() hinzugefügt}
        \begin{leftbar}[0.9\linewidth]
            Wird aufgerufen sobald die Seed Nummer mit Enter bestätigt, oder der Speicher-Button
            rechts daneben geklickt wird.
        \end{leftbar}
\end{itemize}

\subsubsection{SettingsmenuScreen}
\begin{itemize}
    \item \textit{Implementiert nicht mehr ScreenController, erbt dafür von OwnScreenController}
        \begin{leftbar}[0.9\linewidth]
            Es wurde eine eigene ScreenController Klasse hinzugefügt.
        \end{leftbar}
    \item \textit{Methode editCurrentResolution(String, DropDownSelectionChangedEvent) hinzugefügt}
        \begin{leftbar}[0.9\linewidth]
            Wird aufgerufen sobald eine Auswahl im Auflösungs-DropDown getroffen wird.
        \end{leftbar}
    \item \textit{Methode editCurrentLanguage(String, DropDownSelectionChangedEvent) hinzugefügt}
        \begin{leftbar}[0.9\linewidth]
            Wird aufgerufen sobald eine Auswahl im Sprache-DropDown getroffen wird.
        \end{leftbar}
    \item \textit{Methode editCurrentKIDifficulty(String, DropDownSelectionChangedEvent) hinzugefügt}
        \begin{leftbar}[0.9\linewidth]
            Wird aufgerufen sobald eine Auswahl im KI-Schwiriegskeits-DropDown getroffen wird.
        \end{leftbar}
    \item \textit{Methode editCurrentQuality(String, DropDownSelectionChangedEvent) hinzugefügt}
        \begin{leftbar}[0.9\linewidth]
            Wird aufgerufen sobald eine Auswahl im Qualitäts-DropDown getroffen wird.
        \end{leftbar}
    \item \textit{Methode toggleFullscreen(String, CheckBoxStateChangedEvent) hinzugefügt}
        \begin{leftbar}[0.9\linewidth]
            Wird aufgerufen sobald eine Auswahl in der Vollbild-Checkbox getroffen wird.
        \end{leftbar}
    \item \textit{Methode showGraphicsTab() hinzugefügt}
        \begin{leftbar}[0.9\linewidth]
            Zeigt den Grafik Einstellungs-Tab an.
        \end{leftbar}
    \item \textit{Methode showInternTab() hinzugefügt}
        \begin{leftbar}[0.9\linewidth]
            Zeigt den Internen Einstellungs-Tab an.
        \end{leftbar}
    \item \textit{Methode resetSettings() hinzugefügt}
        \begin{leftbar}[0.9\linewidth]
            Setzt die Einstellungen zurück.
        \end{leftbar}
\end{itemize}

\subsubsection{LoadingScreen}
\begin{itemize}
    \item \textit{Implementiert nicht mehr ScreenController, erbt dafür von OwnScreenController}
        \begin{leftbar}[0.9\linewidth]
            Es wurde eine eigene ScreenController Klasse hinzugefügt.
        \end{leftbar}
    \item \textit{Folgende Methoden entfernt}
        \begin{itemize}
            \item openHUD() : HUD wird an anderer Stelle geöffnet.
            \item generateMap() : Funktionalität wude in startGame() verschoben.
            \item getMapBody() : Nicht mehr benötigt.
        \end{itemize}
\end{itemize}

\subsubsection{LoadingUtil}
\begin{itemize}
    \item \textit{Klasse hinzugefügt}
        \begin{leftbar}[0.9\linewidth]
            In dieser Util Klasse existieren Funktionen, die vom LoadingScreen aufgerufen 
            werden, um alle benötigten Sachen zu laden. Mittels Callbacks wird der 
            ladebalken entsprechend gesetzt.
        \end{leftbar}
\end{itemize}

\subsubsection{Hud}
\begin{itemize}
    \item \textit{Implementiert nicht mehr ScreenController, erbt dafür von OwnScreenController}
        \begin{leftbar}[0.9\linewidth]
            Es wurde eine eigene ScreenController Klasse hinzugefügt.
        \end{leftbar}
    \item \textit{Alle Funktionen entfernt/ersetzt}
        \begin{leftbar}[0.9\linewidth]
            Die Minimap wurde nicht im Hud integriert und die restlichen Funktionen wurden 
            ersetzt bzw entfernt. Funktionen um auf andere Bildschirme du wechseln wurde an 
            eine andere Stelle verlagert.
        \end{leftbar}
    \pagebreak
    \item \textit{Methode updateRank(int) hinzugefügt}
        \begin{leftbar}[0.9\linewidth]
            Aktualisiert den angezeigten Rang im Hud.
        \end{leftbar}
    \item \textit{Methode updateTime(float) hinzugefügt}
        \begin{leftbar}[0.9\linewidth]
            Aktualisiert die angezeigte Zeit im Hud.
        \end{leftbar}
    \item \textit{Methode updateCountdown(String) hinzugefügt}
        \begin{leftbar}[0.9\linewidth]
            Aktualisiert den angezeigten Countdown im Hud.
        \end{leftbar}
    \item \textit{Methode updateItem(String) hinzugefügt}
        \begin{leftbar}[0.9\linewidth]
            Aktualisiert das angezeigte Item im Hud.
        \end{leftbar}
\end{itemize}

\subsubsection{ErrorScreen}
\begin{itemize}
    \item \textit{Klasse hinzugefügt}
        \begin{leftbar}[0.9\linewidth]
            Diese Klasse dient dazu, eventuell auftretende Exceptions/Fehler während dem Spiel 
            anzeigen zu können mittels eines Popups.
        \end{leftbar}
\end{itemize}

\subsubsection{PausemenuScreen}
\begin{itemize}
    \item \textit{Implementiert nicht mehr ScreenController, erbt dafür von OwnScreenController}
        \begin{leftbar}[0.9\linewidth]
            Es wurde eine eigene ScreenController Klasse hinzugefügt.
        \end{leftbar}
\end{itemize}

\subsubsection{EndmenuScreen}
\begin{itemize}
    \item \textit{Implementiert nicht mehr ScreenController, erbt dafür von OwnScreenController}
        \begin{leftbar}[0.9\linewidth]
            Es wurde eine eigene ScreenController Klasse hinzugefügt.
        \end{leftbar}
    \item \textit{Methode updatePlayerEndTime() entfernt}
        \begin{leftbar}[0.9\linewidth]
            Ersetzt durch folgende Methode.
        \end{leftbar}
    \item \textit{Methode updatePlayerEndTimeAndColor(int, float, String) hinzugefügt}
        \begin{leftbar}[0.9\linewidth]
            Aktualisiert die Zeit und Farbe des Spielers.
        \end{leftbar}
\end{itemize}