\subsection{Assets}

\subsubsection{ArchitekturÄnderung}
Das Asset-Modul soll in der Lage sein AssetInfos und AssetPacks aus einer Vielzahl von Datei-Formaten
zu lesen. Der Vorschlag des Entwurfs benötigt dafür mehrere AbstractAssetProvider, was zur Laufzeit zu
Problemen führt, da bekannt sein muss, in welchem Format ein Asset gespeichert ist. Um dieses Problem zu
umgehen und das Laden von Assets zu zentralisieren wurden InfoLoader eingeführt, welche AssetInfos und
AssetPacks für einen AssetProvider aus verschiedenen Datei-Formaten lesen, wodurch zur Laufzeit nun
lediglich ein einzelner AssetProvider existieren muss.

\subsubsection{AbstractAssetProvider}
\begin{itemize}
    \item \textit{Klasse entfernt}
        \begin{leftbar}[0.9\linewidth]
            AbstractAssetProvider wurde aufgrund von Änderungen der Architektur entfernt.
        \end{leftbar}
\end{itemize}

\subsubsection{JsonAssetProvider}
\begin{itemize}
    \item \textit{Klasse entfernt}
        \begin{leftbar}[0.9\linewidth]
            JsonAssetProvider wurde aufgrund von Änderungen der Architektur entfernt.
        \end{leftbar}
\end{itemize}

\subsubsection{AssetProvider}
\begin{itemize}
    \item \textit{Klasse hinzugefügt}
        \begin{leftbar}[0.9\linewidth]
            AssetProvider ist für das laden von Assets anhand von AssetPacks und AssetInfos
            zuständig. Diese erhält er von InfoLoadern welche zur Laufzeit dynamisch für
            verschiedene Datei-Typen definiert werden können.
            Der AssetProvider ist ausserdem in der Lage alles asynchron auf einem anderen Thread
            auszuführen.
        \end{leftbar}
\end{itemize}

\subsubsection{AssetContainer}
\begin{itemize}
    \item \textit{Klasse hinzugefügt}
        \begin{leftbar}[0.9\linewidth]
            AssetContainer enthalten den Sub-Scenegraph des geladenen Assets und die dazugehörige AssetInfo
            um neue Instanzen des Assets bereit zu stellen.
        \end{leftbar}
\end{itemize}

\pagebreak

\subsubsection{InfoLoader}
\begin{itemize}
    \item \textit{Interface hinzugefügt}
        \begin{leftbar}[0.9\linewidth]
            InfoLoader wurden im Zuge der Änderungen der Architektur des AssetModuls eingeführt.
            Das interface beschreibt die minimale Funktionalität, welche benötigt wird um mit dem
            AssetProvider zu arbeiten.
        \end{leftbar}
\end{itemize}

\subsubsection{JsonInfoLoader}
\begin{itemize}
    \item \textit{Klasse hinzugefügt}
        \begin{leftbar}[0.9\linewidth]
            Der JsonInfoLoader ist in der Lage AssetPacks und AssetInfos aus Json-Dateien zu lesen.
        \end{leftbar}
\end{itemize}

\subsubsection{MeshInfo}
\begin{itemize}
    \item \textit{Klasse hinzugefügt}
        \begin{leftbar}[0.9\linewidth]
            MeshInfo enthält daten über einen Mesh im Sub-Scenegraph des Assets um diesem die richtigen
            Materialien zu zu weisen.
        \end{leftbar}
\end{itemize}

\subsubsection{AnimationInfo}
\begin{itemize}
    \item \textit{Klasse hinzugefügt}
        \begin{leftbar}[0.9\linewidth]
            AnimationInfo enthält Informationen über Animationen welche auf Objekten im Sub-Scenegraph des
            Assets ausgeführt werden können. Da Animationen selbst im moment nicht ausgeführt werden können
            wird diese Information momentan lediglich genutzt um zu entscheiden ob Assets eines bestimmten
            Typs sich die Meshes teilen können oder nicht.
        \end{leftbar}
\end{itemize}