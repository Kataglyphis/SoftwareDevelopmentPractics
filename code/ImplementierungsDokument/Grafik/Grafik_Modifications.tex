\subsection{InGame - Grafik}

\subsubsection{Partikel}
\begin{itemize}
    \item \textit{Darstellung eines Partikels}
        \begin{leftbar}[0.9\linewidth]
            Partikel hat nun einen Transform, der Rotation, Translation, Größe
            des Partikels bündelt.
        \end{leftbar}
\end{itemize}

\subsubsection{Shader und MaterialDef}
\begin{itemize}
    \item \textit{Wechsel der BRDF}
        \begin{leftbar}[0.9\linewidth]
            Disney's principled BRDF hat an Kanten kleine weiße Pixel produziert.
            Wechsel zum Ansatz, der die Unreal Engine macht, um Problem zu umgehen.
        \end{leftbar}
\end{itemize}

\subsubsection{PartikelEffekt}
\begin{itemize}
    \item \textit{Farbe als Vertexattribut}
        \begin{leftbar}[0.9\linewidth]
        Farbe nicht mehr als Vertexattribut sondern mit übergebenen Ramp Factor berechnet im Shader
        \end{leftbar}
    \item \textit{Partikel töten}
        \begin{leftbar}[0.9\linewidth]
        Ist jetzt Sache der Strategie.
        \end{leftbar}
\end{itemize}

\subsubsection{ParticleStrategy}
\begin{itemize}
    \item \textit{emitParticle(Particle particle, float tpf)}
    \begin{leftbar}[0.9\linewidth]]
        Emitierung von einem Partikel
        Methode behandelt einen einzelnen Partikel.
        Rückgabetyp ist nun void da er nicht weiter benötigt wird.
        Allerdings übergeben wir nun ein Particle und für entsprechende
        Geschwindigkeitsberechnungen tpf(=times per frame)
    \end{leftbar}

    \item \textit{getMaterial()}
    \begin{leftbar}[0.9\linewidth]
        wurde hinzugefügt, da es der particle effect zur Einrichtung benötigt.
    \end{leftbar}

    \item \textit{setNumParticles(int numberOfParticles)}
    \begin{leftbar}[0.9\linewidth]
        wurde hinzugefügt, da es der particle effect zur Einrichtung benötigt.
    \end{leftbar}

    \item \textit{setLifeTime(float lifeTime)}
    \begin{leftbar}[0.9\linewidth]
        wurde hinzugefügt, da der Partikel PartikelEffekt
    \end{leftbar}

    \item \textit{ void initializeParticleData(Particle[] particles)}
    \begin{leftbar}[0.9\linewidth]
        wurde hinzugefügt, da Strategie partikel initialisieren muss
    \end{leftbar}
\end{itemize}

\subsubsection{ViewPortManager}
\begin{itemize}
    \item \textit{Klasse hinzugefügt}
    \begin{leftbar}[0.9\linewidth]
        ViewPortManager bietet ein einfaches Interface zum erstellen von ViewPorts für
        verschiedene Verwendungszwecke. Eingeführt, um die Applikationsarchitektur mehr
        auf AppStates basieren zu lassen.
    \end{leftbar}
\end{itemize}

\subsubsection{SceneManager}
\begin{itemize}
    \item \textit{Klasse hinzugefügt}
    \begin{leftbar}[0.9\linewidth]
        SceneManager ist der Haupt-Appstate welcher während dem Rennen aktiv ist.
        Er ist dafür verantwortlich alle ViewPorts (für SplitScreen) zu initialisieren
        und später auch wieder zu zerstören, die CullingManager zu erstellen und die gesammte
        Szene zusammen zu führen.
    \end{leftbar}
\end{itemize}

\subsubsection{CullingManager}
\begin{itemize}
    \item \textit{Klasse hinzugefügt}
    \begin{leftbar}[0.9\linewidth]
        CullingManager ist ein SceneProcessor, welcher dafür sorgt,
        dass in einem ViewPort nicht mehr Geometrie gerendert wird als notwendig.
        Wurde hinzugefügt um Performance zu verbessern.
    \end{leftbar}
\end{itemize}

\subsubsection{CullingControl}
\begin{itemize}
    \item \textit{Klasse hinzugefügt}
    \begin{leftbar}[0.9\linewidth]
        CullingControl ist an jedem Spatial angehängt, was vom CullingManager kontrolliert werden kann.
        Wurde hinzugefügt um das verstecken von Spatials im Scenegraph zu vereinfachen und um dem
        CullingManager einfachen Zugriff auf nötige Informationen zu geben.
    \end{leftbar}
\end{itemize}

\subsubsection{CullableRegister}
\begin{itemize}
    \item \textit{Klasse hinzugefügt}
    \begin{leftbar}[0.9\linewidth]
        CullableRegister enthält alle Spatials, welche ein CullingControl beinhalten.
        Wurde hinzugefügt um Zugriff auf die statische Szene zu zentralisieren, da es mehrere
        CullingManager geben kann (sofern SplitScreen aktiv ist).
    \end{leftbar}
\end{itemize}