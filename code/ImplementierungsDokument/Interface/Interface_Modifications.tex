\subsection{InGame - Interface}

\subsubsection{DynamicGameObjectType}
\begin{itemize}
    \item \textit{Enum hinzugefügt}
        \begin{leftbar}[0.9\linewidth]
            Wird verwendet um DynamicGameObjects zu identifizieren.
            Wurde auf Hinweis des anderen Teams hinzugefügt um das
            klassifizieren von DynamicGameObjects zu vereinfachen.
        \end{leftbar}
\end{itemize}

\subsubsection{DynamicGameObject}
\begin{itemize}
    \item \textit{Attribut für DynamicGameObjectType hinzugefügt}
        \begin{leftbar}[0.9\linewidth]
            Enthält einen DynamicGameObjectType welcher während der Lebenszeit eines DynamicGameObjects
            nicht verändert werden kann. Wurde auf Hinweis des anderen Teams implementiert um das
            klassifizieren von DynamicGameObjects zu vereinfachen.
        \end{leftbar}

    \item \textit{Methode addProperty(IProperty, int) hinzugefügt}
        \begin{leftbar}[0.9\linewidth]
            Fügt eine neue IProperty hinzu, aber berücksichtigt freie Ids nur wenn sie mindestens so groß
            wie der gegebene Index. Wurde hinzugefügt um das finden bestimmter IProperties zur Laufzeit
            zu vereinfachen.
        \end{leftbar}

    \item \textit{Methode setProperty(int, IProperty) hinzugefügt}
        \begin{leftbar}[0.9\linewidth]
            Setzt die IProperty an dem gegebenen Index, falls dieser existiert.
            Wurde hinzugefügt um das Aktualisieren der IProperties im Simulationsthread zu erleichtern.
        \end{leftbar}
\end{itemize}

\subsubsection{Tick}
\begin{itemize}
    \item \textit{Signatur von getDynamicGameObjects() geändert}
        \begin{leftbar}[0.9\linewidth]
            Die Methode nimmt nun zusätzlich einen Boolean entgegen um zu entscheiden, ob das Resultat die
            Ids der DynamicGameObjects erhalten soll oder ob NULL elemente aus der Liste entfernt werden
            sollen.
        \end{leftbar}

    \item \textit{Methode removeDynamicGameObject(DynamicGameObject) hinzugefügt}
        \begin{leftbar}[0.9\linewidth]
            Entfernt das gegebene DynamicGameObject aus dem Tick, sofern es existiert.
            Wurde auf Wunsch des anderen Teams implementiert um das Management von DynamicGameObjects
            im Tick zu vereinfachen.
        \end{leftbar}
\end{itemize}