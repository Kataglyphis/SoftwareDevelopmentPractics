\subsection{Generierung}

\subsubsection{IColoring}
\begin{itemize}
    \item \textit{Schnittstelle IColoring erstellt}
        \begin{leftbar}[0.9\linewidth]
            Definiert die nötige Funktionalität für eine Färbungsstrategie, die einem Dome mitgegeben werden kann.
        \end{leftbar}
    
\end{itemize}

\subsubsection{BiomMeadowColoring}
\begin{itemize}
    \item \textit{Erste konkrete Strategie zum Färben}
        \begin{leftbar}[0.9\linewidth]
            Wird verwendet, um ein Biom vom Typ 'BiomMeadow' einzufärben.
        \end{leftbar}
    
\end{itemize}

\subsubsection{triangulation}
\begin{itemize}
    \item \textit{Package von GitHub importiert}
        \begin{leftbar}[0.9\linewidth]
            Wird verwendet, um das Gelände eines Domes dem 'low-ploly' Stil anzupassen.
        \end{leftbar}
    
\end{itemize}


\subsubsection{GridVertex}
\begin{itemize}
    \item \textit{Attribut Color color entfernt}
        \begin{leftbar}[0.9\linewidth]
            Color wird im GridVertex nicht mehr benötigt, da im neuen \textit{IColoring} direkt ein colorArray, abhängig
            von den \textit{GridVertexProperty}, erzeugt wird und als VertexAttribut gesetzt wird.
        \end{leftbar}
\end{itemize}

\subsubsection{CullingManager}
\begin{itemize}
    \item \textit{Klasse wurde gelöscht}
        \begin{leftbar}[0.9\linewidth]
            \textit{CullingManager} wurde in \textit{Rendering} verschoben.
        \end{leftbar}
\end{itemize}

\subsubsection{MapGenerator}
\begin{itemize}
    \item \textit{Signatur des Konstruktors MapGenerator(String confPath, AssetManager assetManager) geändert}
        \begin{leftbar}[0.9\linewidth]
            Der \textit{confPath} wurde entfernt, da nun ein Objekt des Typs \textit{GenerationConfig} erzeugt wird
            und der \textit{paramPath} entsprechend gesetzt wird, der immer gleich bleibt.\par

            Der \textit{appStateManager} wird bnötigt um Zugriff auf den AssetProvider zu erhalten.\par

            Der Parameter \textit{vehileCount} wird benötigt, um alle StartPositionen für Fahrzeuge festzulegen.
        \end{leftbar}
\end{itemize}

\subsubsection{Map}
\begin{itemize}
    \item \textit{Signatur dees Konstruktors Map(appStateManager: AppStateManager) geändert}
        \begin{leftbar}[0.9\linewidth]
            Der \textit{appStateManager} sollte wegen des CullingManagers übergegben werde, da dieser verschoben wurde
            ist auch der appStateManager überflüssig geworden.\par

            Es wird nun ein \textit{AssetManager} übergeben, um das Straßen Model laden zu können.
        \end{leftbar}
        
    \item \textit{Signatur von addSceneItem(ISceneItem sceneItem, int cullingIndex)  geändert}
        \begin{leftbar}[0.9\linewidth]
            Der Paramter \textit{cullingIndex} wird wegen des neuen CullingManagers nicht mehr benötigt.
        \end{leftbar}
        
    \item \textit{Signatur von generateSceneGraph(): Node  geändert}
        \begin{leftbar}[0.9\linewidth]
            Die Methode gibt nun keine Node mehr zurück, sondern bekommt eine \textit{rootNode} als Parameter, an die alle 
            sceneItems gehängt werden, da der CullingManager die sceneItem direkt an der rootNode erwartet.
        \end{leftbar}
\end{itemize}

\subsubsection{IMapBody}
\begin{itemize}
    \item \textit{Signatur von calc3DCoordinates(Vector2f coordinates2D, int index): Vector3f geändert}
        \begin{leftbar}[0.9\linewidth]
            Die Methode wurde umbenannt in calc3DTransform und gibt nun ein Transform statt des Vector3f zurück.\par

            Statt des Vektor2f wird nun eine Position übergeben.\par

            Der \textit{index} wird nicht mehr benötigt, stattdessen wird ein \textit{angle} mitgegeben.
        \end{leftbar}
\end{itemize}

\subsubsection{DomeGenerator}
\begin{itemize}
    \item \textit{Attribut paramPath wurde hinzugefügt}
        \begin{leftbar}[0.9\linewidth]
            Das Attribut wurde erstellt um die Navigation in der \textit{GenerationConfig.json} übersichtlicher zu halten.
        \end{leftbar}
\end{itemize}

\subsubsection{AbstractNoiseGenerator}
\begin{itemize}
    \item \textit{Attribut paramPath hinzugefügt und biom gelöscht}
    \begin{leftbar}[0.9\linewidth]
            paramPath wurde erstellt um die Navigation in der \textit{GenerationConfig.json} übersichtlicher zu halten.\par
            
            Das biom wird jetzt in \textit{makeNoise} mitgegeben und wird deshalb nicht mehr benötigt.
        \end{leftbar}
    
        \item \textit{Konstruktor SimplexNoisGenerator(GenerationConfig generationConfig, AssetManager assetManager, String biom) verändert}
        \begin{leftbar}[0.9\linewidth]
            Der assetManager wurde entfernt, da er nicht gebraucht wird.\par
            
            Das biom wir nicht mehr übergeben, da dies nun in \textit{makeNoise} gemacht wird.
        \end{leftbar}
    \end{itemize}
    
\subsubsection{OpenSimplexNoise}
\begin{itemize}
    \item \textit{Sinatur von  makeNoise(HashMap<GridVertex> grid, int seed) geändert}
        \begin{leftbar}[0.9\linewidth]
            Der Parameter \textit{int seed} wurde entfernt, da dieser im Konstruktor benötigt wird.
            Zusätzlich wurde \textit{String biom} hinzugefügt, um Parameter aus der \textit{GenerationConfig.json}
            auslesen zu können.
        \end{leftbar}
\end{itemize}
    
\subsubsection{AbstractDomeAssetGenerator}
\begin{itemize}
    \item \textit{Wurde um einen Konstruktor ergänzt}
        \begin{leftbar}[0.9\linewidth]
            Der Konstruktor wurde erstellt, da dieser für Funktionalität benötigt wird und Unterklassen diesen einfach verwenden können,
            damit es nicht zu Problemen kommt.
        \end{leftbar}

    \item \textit{Die Methode generateObjects(HashMap<GridVertex> grid, Road road, Node domeAssetsRootNode, int seed) wurde entfernt}
        \begin{leftbar}[0.9\linewidth]
            Die Shablonenmethode wurde entfernt, damit ein DomeGenerator selbst entscheiden kann, welche Funktionalität er in Anspruch nimmt.
        \end{leftbar}
    
\end{itemize}

\subsubsection{DomeAssetGenerator}
\begin{itemize}
    \item \textit{Der Konstruktor DomeObjectGenerator(GenerationConfig generationConfig,String biom, AssetManager assetManager) wurde verändert}
        \begin{leftbar}[0.9\linewidth]
            Der Konstruktor wurde dem des AbstractDomeAssetGenerator angepasst.\par

            Das biom wird somit nicht mehr übergeben, da dieses in den Methoden als Parameter übergeben wird.
        \end{leftbar}
    
\end{itemize}

\subsubsection{Dome}
\begin{itemize}
    \item \textit{Neue Attribute wurden hinzugefügt}
        \begin{leftbar}[0.9\linewidth]
            Die landscapeResolution bestimmt , wie viele GridVertecis in ein Dreieck der Map aufgenommen werden, das ist für die Triangulation
            nötig.\par

            Das material bestimmt das Material für den gesammten Dome.\par

            Das coloring wird verwendet, um eine Strategie für das Färben der Landschaft zu setzen.
        \end{leftbar}

    \item \textit{Der Konstruktor  Dome(HashMap<GridVertex> gird, Road road, Node domeAssetsRootNode) wurde erweitert}
        \begin{leftbar}[0.9\linewidth]
            Konstruktor wurde an die neuen Attribute angepasst.
        \end{leftbar}
    
\end{itemize}







\subsubsection{ISceneItem}
\begin{itemize}
    \item \textit{Methoden hinzugefügt}
        \begin{leftbar}[0.9\linewidth]
            Die Methoden getEntryFram(), getExitFrame wurden hinzugefügt. Diese geben den Querschnitt der Road am ersten und letzten
            RoadCursor an. Diese Information ist für den CullingManager nötig.\par
        \end{leftbar}
\end{itemize}


\subsubsection{ITunnelGenerator}
\begin{itemize}
    \item \textit{Methodenschnittstelle verändert: ITunnel generate(int seed, Vector3f targetPosition, float horizontalAngle, float verticalAngle,
    Vector2f entryWidth, Set<GeneratorSettings> flags)}
        \begin{leftbar}[0.9\linewidth]
            Zielpunkt wird durch eine Position und zwei Winkel bestimmt anstatt, durch einen RoadCursor.\par
        \end{leftbar}
\end{itemize}

\subsubsection{TunnelGenerator}
\begin{itemize}
    \item \textit{Konstruktorschnittstelle verändert: TunnelGenerator(GenerationConfig generationConfig, AssetProvider assetProvider)}
        \begin{leftbar}[0.9\linewidth]
            AssetManager wird AssetProvider ausgetauscht.\par
        \end{leftbar}
\end{itemize}


\subsubsection{Tunnel}
\begin{itemize}
    \item \textit{Konstruktorschnittstelle verändert: Tunnel(Road road, Node tunnelAssetRootNode)}
        \begin{leftbar}[0.9\linewidth]
            HashMap von GridVertex verworfen\par
        \end{leftbar}
\end{itemize}

\subsubsection{RoomSelector}
\begin{itemize}
    \item \textit{Konstruktorschnittstelle verändert: RoomSelector(GenerationConfig generationConfig, AssetProvider assetProvider, int vehicleCount)}
        \begin{leftbar}[0.9\linewidth]
            AssetManager durch AssetProvider ausgetauscht, Fahrzeugeanzahl hinzugefüht\par
        \end{leftbar}
\end{itemize}


\subsubsection{DynamicRoomGenerator}
\begin{itemize}
    \item \textit{Konstruktorschnittstelle verändert: DynamicRoomGenerator(GenerationConfig generationConfig, AssetProvider assetProvider, int vehicleCount)}
        \begin{leftbar}[0.9\linewidth]
            AssetManager durch AssetProvider ausgetauscht, Fahrzeugeanzahl hinzugefüht\par
        \end{leftbar}
\end{itemize}


\subsubsection{StaticRoomGenerator}
\begin{itemize}
    \item \textit{Klasse nicht implementiert}
        \begin{leftbar}[0.9\linewidth]
            Alle Räume werden durch den DynamicRoomGenerator generiert.\par
        \end{leftbar}
\end{itemize}

\subsubsection{StaticRoom}
\begin{itemize}
    \item \textit{Klasse nicht implementiert}
        \begin{leftbar}[0.9\linewidth]
            Alle Räume sind durch den DynamicRoomGenerator generierte DynamicRooms.\par
        \end{leftbar}
\end{itemize}

\subsubsection{DynamicRoom}
\begin{itemize}
    \item \textit{Konstruktorschnittstelle verändert: DynamicRoom(Road road, Node assetRootNode)}
        \begin{leftbar}[0.9\linewidth]
            HashMap von GridVertex verworfen\par
        \end{leftbar}
\end{itemize}


\subsubsection{IRoadGenerator}
\begin{itemize}
    \item \textit{Methodenschnittstellen verändert: Road generateSphereRoad(int seed, float radius, String biom, Vector2f entryWidth, float exitAngle,
    Set<GeneratorSettings> flags)}
        \begin{leftbar}[0.9\linewidth]
            Set<GeneratorSettings> anstatt Collection<GeneratorSettings>\par
        \end{leftbar}
    \item \textit{Methodenschnittstelle verändert: Road generateConnectingRoad(int seed, RoadCursor target, Vector2f entryWidth,
    String settings, Set<GeneratorSettings> flags)}
        \begin{leftbar}[0.9\linewidth]
            Set<GeneratorSettings> anstatt Collection<GeneratorSettings> sowie String setting hinzugefügt,
            welche die Konfiguration zur Streckengenerierung auswählt.\par
        \end{leftbar}
    \item \textit{Methodenschnittstelle verändert: Road generateCuboidRoad(int seed, Vector3f cuboidDimensions, Vector2f entryWidth, float exitAngle,
    String settings, Set<GeneratorSettings> flags)}
        \begin{leftbar}[0.9\linewidth]
            Set<GeneratorSettings> anstatt Collection<GeneratorSettings> sowie String setting hinzugefügt,
            welche die Konfiguration zur Streckengenerierung auswählt.\par
        \end{leftbar}
\end{itemize}

\subsubsection{RoadCursor}
\begin{itemize}
    \item \textit{Methoden aufgeteilt}
        \begin{leftbar}[0.9\linewidth]
            getAngle() in 3 separate Methoden für X,Y,Z Winkle aufgeteilt.\par
        \end{leftbar}getRoadFrame()
    \item \textit{Methoden hinzugefügt: getRoadFrame()}
        \begin{leftbar}[0.9\linewidth]
            Gibt 4 Punkte zurück welche die Querschnittsfläche der Strecke in diesem RoadCursor beschreiben.\par
        \end{leftbar}
\end{itemize}

\subsubsection{RoadObject}
\begin{itemize}
    \item \textit{Konstruktorschnittstelle verändert: RoadObject(Vector3f position, int index)}
        \begin{leftbar}[0.9\linewidth]
            Index des RoadCursors hinzugefügt\par
        \end{leftbar}
    \item \textit{Methoden hinzugefügts}
        \begin{leftbar}[0.9\linewidth]
            Getter und setter für Index\par
        \end{leftbar}
\end{itemize}

\subsubsection{MathUtility - Klasse hinzugefügt}
Klasse zur Bereitstellung Mathematischer operationen.
\begin{itemize}
    \item \textit{Methode : public static Vector3f rotateVectorCC(Vector3f vec, Vector3f axis, double theta)}
        \begin{leftbar}[0.9\linewidth]
            Rotiert Vektor um Achse (Quelle: StackOverlfow)\par
        \end{leftbar}
    \item \textit{Methoden : public static float limitAbsValue(float value, float limit)}
        \begin{leftbar}[0.9\linewidth]
            Begrenzt absuluten Wert von value\par
        \end{leftbar}
\end{itemize}


\subsubsection{AssetsUtility - Klasse hinzugefügt}
Klasse zur Bereitstellung von Operationen welche zur Transformation von Assets bereitstellt.
\begin{itemize}
    \item \textit{Methode : public static Node generateRoadAsset(Road road, Spatial asset, Vector3f assetDimensions, int segmentsPerAsset, boolean scaleOnY)}
        \begin{leftbar}[0.9\linewidth]
            Skaliert ein Asset über die Länge der Strecke.\par
        \end{leftbar}
    \item \textit{Methoden : public static void setLightToRoot(Node rootNode, Node parentNode)}
        \begin{leftbar}[0.9\linewidth]
            Hängt rekursiv alle Lichter des Szenen Grafs an die rootNode.\par
        \end{leftbar}
    \item \textit{Methoden : public static Node tranformAssetAlongRoadCursor(RoadCursor roadCursor, Spatial asset,
        Vector3f assetDimensions, boolean scaleOnY)}
        \begin{leftbar}[0.9\linewidth]
            Skaliert ein Asset entsprechend der Breite und Höhe der Strecke in gegebenem RoadCursor.\par
        \end{leftbar}
    \item \textit{Methoden : public static Spatial transformBetweenPoints(Vector3f pointA, Vector3f pointB,
    Vector3f pointC, Vector3f pointD, Spatial asset, Vector3f assetDimensions)}
        \begin{leftbar}[0.9\linewidth]
            Skaliert ein Asset Zwischen vier Punkten.\par
        \end{leftbar}
\end{itemize}