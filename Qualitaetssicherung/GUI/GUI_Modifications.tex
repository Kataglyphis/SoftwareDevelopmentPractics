\subsection{GUI}

\subsubsection{Neue Features}
\begin{itemize}
    \item \textit{Ladebalken kleinschrittiger}
        \begin{leftbar}[0.9\linewidth]
            Der Ladebalken ist nun kleinschrittiger, sobald die 
            Streckengenerierung startet.
        \end{leftbar}
    \item \textit{Countdown Anzeige}
        \begin{leftbar}[0.9\linewidth]
            Es wird nun ein Countdown Timer am Anfang des Rennens 
            angezeigt. Sobald dieser abgelaufen ist, startet das 
            Rennen zusammen mit der Zeit-Anzeige unten links.
        \end{leftbar}
    \item \textit{Anzeige einer Console}
        \begin{leftbar}[0.9\linewidth]
            Wird die Taste F5 gedrückt, erscheint eine Console mit 
            welcher man eine Simulation starten kann, wo mit 
            der FlyCam die Spielewelt erkundet werden kann.
            Außerdem lässt sich ein Profiler anzeigen, welcher 
            eine Fps Angabe einblendet.
        \end{leftbar}
    \item \textit{Möglichkeit, die Spieler Anzahl zu bearbeiten}
        \begin{leftbar}[0.9\linewidth]
            Bei der Streckenauswahl lässt nun zusätzlich zu der Anzahl
            KI's die Anzahl an Spielern festlegen. Zur Zeit ist dies
            jedoch nicht bearbeitbar, da der Mehrspielermodus 
            zusammen mit dem anderen Team nicht weiter ausgebaut 
            werden konnte.
        \end{leftbar}
    \item \textit{Streckentyp auswählbar}
        \begin{leftbar}[0.9\linewidth]
            Die Angabe des Streckentyps ermöglicht eine kurze,
            normale oder lange Strecke. Zukunftsorientiert sind 
            hier noch weitere Typen möglich.
        \end{leftbar}
    \item \textit{Credits}
        \begin{leftbar}[0.9\linewidth]
            Es besteht die Möglichkeit, sich im Startmenü Credits 
            anzeigen zu lassen.
        \end{leftbar}
    \item \textit{Tastaturbelegung}
        \begin{leftbar}[0.9\linewidth]
            Die Tastaturbelegung wird nun im Einstellungsmenü, 
            unter dem Logik-Tab angezeigt, ist aber nicht editierbar.
        \end{leftbar}
\end{itemize}

\subsubsection{Änderungen existierender Komponenten}
\begin{itemize}
    \item \textit{Allgemeine Menü-Aussehen}
        \begin{leftbar}[0.9\linewidth]
            Die Menüs wurden vom Aussehen her geupdatet. Unter anderem
            wurde der Hintergrund ersetzt und einige Elemente
            aktualisiert.
        \end{leftbar}
    \item \textit{Item-Anzeige}
        \begin{leftbar}[0.9\linewidth]
            Anstatt Text werden Items jetzt durch Icons visualisiert.
        \end{leftbar}
\end{itemize}

\subsubsection{Nicht implementiert}
\begin{itemize}
    \item \textit{Auswählen des Spielmodus}
        \begin{leftbar}[0.9\linewidth]
            Wurde nicht weiter erörtert, da der Mehrspielermodus 
            seitens des Physik-Teams leider ausfallen musste.
        \end{leftbar}
    \item \textit{Mini-Map}
        \begin{leftbar}[0.9\linewidth]
            Wurde aufgrund mangelnder Zeit nicht implementiert.
            Als Team wurde entschieden, anderen Kriterien mehr 
            Fokus zu geben.
        \end{leftbar}
    \item \textit{Rundenzahl}
        \begin{leftbar}[0.9\linewidth]
            Da es keine Runden gibt, gibt es entsprechend auch keine 
            Rundenzahl.
        \end{leftbar}
    \item \textit{Fahrergeschwindigkeit}
        \begin{leftbar}[0.9\linewidth]
            Wird nicht im HUD direkt angezeigt, sondern am 
            Fahrzeug des Spielers.
        \end{leftbar}
\end{itemize}