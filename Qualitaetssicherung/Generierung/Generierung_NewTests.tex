\subsection{Generierung}

\subsubsection{Automatische Tests}
\paragraph{BiomMeadowColoringTest}
\begin{itemize}
    \item \textit{makeColorRoad}
        \begin{leftbar}[0.9\linewidth]
			Testet die Färbung zweier Dreiecke, bei denen ein Fahrbahnuntergrund zu sehen sein soll.
			Jeder Farbkanal enthält den gleichen Farbwert: 0.1f.
		\end{leftbar}
		
	\item \textit{makeColorRock}
        \begin{leftbar}[0.9\linewidth]
			Testet die Färbung zweier Dreiecke, bei denen ein Stein zu sehen sein soll.
			Jeder Farbkanal enthält den gleichen Wert: 0f.
		\end{leftbar}
		
	\item \textit{makeColorSnow}
        \begin{leftbar}[0.9\linewidth]
            Testet die Färbung zweier Dreiecke, bei denen Schnee zu sehen sein soll.
			Jeder Farbkanal enthält den gleichen Wert: 1f.
		\end{leftbar}
		
	\item \textit{makeColorMeadow}
		\begin{leftbar}[0.9\linewidth]
			Testet die Färbung zweier Dreiecke, bei denen Wiese zu sehen sein soll.
			Jeder Farbkanal enthält unterschiedliche Werte zwischen 0f und 1f.
		\end{leftbar}

	\item \textit{makeColorSnowOverMeadow}
        \begin{leftbar}[0.9\linewidth]
			Testet die Färbung zweier Dreiecke, bei denen nicht alle Eckpunkte die gleichen Eigenschaftenm aufweisen,
			einige Eckpunkte 'bestehen aus' Wiese andere aus Schnee.
			Die Farbkanäle enthalten die oben genannten Werte, je für Schnee bzw. Wiese.
		\end{leftbar}
		
	\item \textit{makeColorMeadowOverRock}
        \begin{leftbar}[0.9\linewidth]
			Testet die Färbung zweier Dreiecke, bei denen nicht alle Eckpunkte die gleichen Eigenschaftenm aufweisen,
			einige Eckpunkte 'bestehen aus' Wiese andere aus Stein.
			Die Farbkanäle enthalten die oben genannten Werte, je für Stein bzw. Wiese.
        \end{leftbar}
\end{itemize}

\paragraph{MapBodyTest}
\begin{itemize}
    \item \textit{test0Calc3DTransform}
        \begin{leftbar}[0.9\linewidth]
			Testet die Transformat von Positionen aus dem RoadModel
			in Weltkoordinaten.
		\end{leftbar}
	\item \textit{test1Calc3DTransform}
        \begin{leftbar}[0.9\linewidth]
            Testet die Transformat von Positionen aus dem RoadModel
			in Weltkoordinaten.
		\end{leftbar}
	\item \textit{test2Calc3DTransform}
        \begin{leftbar}[0.9\linewidth]
            Testet die Transformat von Positionen aus dem RoadModel
			in Weltkoordinaten.
		\end{leftbar}
\end{itemize}

\paragraph{RoadGeneratorTest}
\begin{itemize}
	\item \textit{generateSphereRoad}
        \begin{leftbar}[0.9\linewidth]
			Testet ob die Generierung einer Strecke für eine Kuppel erfolgreich war.
			Testet die Koorektheit der Rotation der RoadCursor Instanzen.
			Testet ob die Minimaldistanz zwischen den RoadCursor Instanzen eingehalten.
		\end{leftbar}
	\item \textit{generateConnectingRoad}
        \begin{leftbar}[0.9\linewidth]
            Testet ob die Generierung einer Strecke für einen Tunnel erfolgreich war.
			Testet die Koorektheit der Rotation der RoadCursor Instanzen.
			Testet ob die Minimaldistanz zwischen den RoadCursor Instanzen eingehalten.
		\end{leftbar}
	\item \textit{generateCuboidRoad}
        \begin{leftbar}[0.9\linewidth]
            Testet ob die Generierung einer Strecke für einen Raum erfolgreich war.
			Testet die Koorektheit der Rotation der RoadCursor Instanzen.
			Testet ob die Minimaldistanz zwischen den RoadCursor Instanzen eingehalten.
		\end{leftbar}
\end{itemize}

\paragraph{DynamicRoomGeneratorTest}
\begin{itemize}
	\item \textit{test0GenerateRoom}
        \begin{leftbar}[0.9\linewidth]
            Testet ob die Raumgenerierung unter bestimmten Parametern erfolgreich ist.
		\end{leftbar}
	\item \textit{test1GenerateRoom}
        \begin{leftbar}[0.9\linewidth]
            Testet ob die Raumgenerierung unter bestimmten Parametern erfolgreich ist.
		\end{leftbar}
	\item \textit{test2GenerateRoom}
        \begin{leftbar}[0.9\linewidth]
            Testet ob die Raumgenerierung unter bestimmten Parametern erfolgreich ist.
		\end{leftbar}
	\item \textit{testGenerateStartRoom}
        \begin{leftbar}[0.9\linewidth]
			Testet ob die Raumgenerierung unter bestimmten Parametern erfolgreich ist.
			Testet ob die Richtige Anzahl an Startpositionen Platziert wurde.
		\end{leftbar}
	\item \textit{testGenerateEndRoom}
        \begin{leftbar}[0.9\linewidth]
			Testet ob die Raumgenerierung unter bestimmten Parametern erfolgreich ist.
			Testet ob die Ziellinie korrekt platziert wurde.
		\end{leftbar}
\end{itemize}


\paragraph{TunnelGeneratorTest}
\begin{itemize}
	\item \textit{testGenerate}
        \begin{leftbar}[0.9\linewidth]
			Testet ob die TunnelGenerierung mit unterschiedlichen Parametern erfolgreich ist.
			Testet ob der letzte RoadCursor in Position und Rotation nur minimal vom "Ziel"-RoadCursor abweicht.
		\end{leftbar}
\end{itemize}


\paragraph{RoadTest}
\begin{itemize}
	\item \textit{testUpdateDirections}
        \begin{leftbar}[0.9\linewidth]
			Testet ob die Methode testUpdateDirections(...) die Rotation der RoadCursor korrekt korrigiert.
		\end{leftbar}
	\item \textit{attach}
        \begin{leftbar}[0.9\linewidth]
			Testet ob die Methode attach(...) die Positionen der RoadCursor Instanzen der
			angehängten Road korrekt transformiert.
		\end{leftbar}
\end{itemize}

\paragraph{RoadCursorTest}
\begin{itemize}
	\item \textit{testPosition}
        \begin{leftbar}[0.9\linewidth]
			Testet die Position der RoadCursor:
			Nach dem Erstellen durch unterschiedliche Konstruktoren,
			Nach Anwendung von Rotation durch die Methode applyRotation(...),
			Nach Anwendung von Transformation durch die Methode applyTransformation(...);
		\end{leftbar}
	\item \textit{testDirection}
        \begin{leftbar}[0.9\linewidth]
			Testet den Richtungsvektor der RoadCursor:
			Nach dem Erstellen durch unterschiedliche Konstruktoren,
			Nach Anwendung von Rotation durch die Methode applyRotation(...),
			Nach Anwendung von Transformation durch die Methode applyTransformation(...);
		\end{leftbar}
	\item \textit{testRight}
        \begin{leftbar}[0.9\linewidth]
			Testet den Rechtsvektor der RoadCursor:
			Nach dem Erstellen durch unterschiedliche Konstruktoren,
			Nach Anwendung von Rotation durch die Methode applyRotation(...),
			Nach Anwendung von Transformation durch die Methode applyTransformation(...);
		\end{leftbar}
	\item \textit{testNormal}
        \begin{leftbar}[0.9\linewidth]
			Testet den Normalenvektor der RoadCursor:
			Nach dem Erstellen durch unterschiedliche Konstruktoren,
			Nach Anwendung von Rotation durch die Methode applyRotation(...),
			Nach Anwendung von Transformation durch die Methode applyTransformation(...);
		\end{leftbar}
	\item \textit{testXZAngle}
        \begin{leftbar}[0.9\linewidth]
			Testet die Berechnung des Winkels in der Ebene der RoadCursor.
		\end{leftbar}
	\item \textit{testYAngle}
        \begin{leftbar}[0.9\linewidth]
			Testet die Berechnung des Steigungs Winkels der RoadCursor.
		\end{leftbar}
	\item \textit{testTiltAngle}
        \begin{leftbar}[0.9\linewidth]
			Testet ob der Kippwinkel des RoadCursors 0 ist.
		\end{leftbar}
	\item \textit{testRoadCursorProperty}
        \begin{leftbar}[0.9\linewidth]
			Testet ob der RoadCursor die ihm zugewiesene RoadCursorProperty hat.
		\end{leftbar}
	\item \textit{testRoadFrame}
        \begin{leftbar}[0.9\linewidth]
			Testet ob die Methode testRoadFrame(...) korrekt funktioniert.
		\end{leftbar}
\end{itemize}

\subsubsection{Manuelle Tests}

	Zusätzlich zu den oben genannten JUnit-Tests wurde zum größten Teil manuell getestet, da optische Eindrücke 
	nicht automatisiert getestet werden können. Hierzu wurde für jeden betrachteten Teil ggf. die 
	\textit{GenerationConfig} kurzzeitig angepasst und dann das Spiel mit der DummySimulation ausgeführt, um so das
	entsprechende Element auf Sicht zu prüfen. Hierzu wurden die betrachteten Teile vorallem möglichst klein gehalten
	und manchmal auch beispielsweise Screenshots als Hilfestellung verwendet. Unten aufgelistet sind beispielhafte 
	Testabläufe, welche noch nicht im Pflichtenheft definiert wurden.

	\begin{itemize} [label={}]
		\item \textbf{ET120}\par %Bitte im Pflichtenheft nachschauen nach der nächsten Nummer
			\textbf{STAND} Benutzer im Hauptmenü.\\
			\textbf{AKTION} Generierung mit beleuchteten Räumen.\\
            \textbf{RESULTAT} Beleuchtung wirkt natürlich.\\\par
            
        \item \textbf{ET130}\par %Bitte im Pflichtenheft nachschauen nach der nächsten Nummer
			\textbf{STAND} Generierung, mit Kuppel und darin leigender Straße, abgeschlossen.\\
			\textbf{AKTION} Betrachtung der auf die Straße angepassten Landschaft.\\
			\textbf{RESULTAT} Landschaft unterhalb der Straße wird, in allen Punkten, 
			genau unter die Straße gesetz.\\\par

	\end{itemize}