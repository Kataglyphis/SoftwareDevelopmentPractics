\subsection{Generierung}

\subsubsection{Neue Features}
\begin{itemize}
    \item \textit{Bepflanzung der Kuppeln}
        \begin{leftbar}[0.9\linewidth]
            Es können nun für ein bestimmtes Biom eine Menge von Bäumen(bzw. ähnliche Pflanzen) definiert werden,
            welche dann in einer entsprechenden Kuppel platziert werden. Bei der Platzierung werden die Objekte
            anhand einer, im \textit{DomeAssetGenerator} definierten, Dichtefunktion verschoben, wie stark 
            kann durch einen Parameter pro Biom bestimmt werden.
            Desweiteren kann man pro Biom unter anderem die Asset-Skalierung und Farben der Blätter festlegen.
        \end{leftbar}

    \item \textit{Assets als Kuppeldach}
        \begin{leftbar}[0.9\linewidth]
            Es ist nun möglich Assets als Dach für eine Kuppel zu verwenden. Es können auch verschiedene Assets
            definiert werden, aus denen dann ein zufälliges(abhängig vom Seed) ausgesucht wird.
        \end{leftbar}
\end{itemize}

\subsubsection{Änderungen existierender Komponenten}
\begin{itemize}
    \item \textit{Methode \textit{generateScenegraph()} parallelisiert}
        \begin{leftbar}[0.9\linewidth]
            \textit{GenerateScenegraph()} der Klasse Map, ruft nun die \textit{generateScenegraph()} Methoden
            der ISceneItem parallelisiert auf und beschleunigt so den Generierungsprozess deutlich.
        \end{leftbar}
    
    \item \textit{'Colring' von Landschaften verschönert}
        \begin{leftbar}[0.9\linewidth]
            Es entstehen nun keine ungewollten (bunte) Dreiecksstrukturen mehr, da das festlegen der 'Oberflächen'
            an einem Punkt optimiert wurde.
        \end{leftbar}

    \item \textit{Verbesserte Räume}
        \begin{leftbar}[0.9\linewidth]
            Ein weiterer Raumtyp wurde hinzugefügt, sowie die Asset-Platzierung in räumen optimiert.
        \end{leftbar}

    \item \textit{Unterschiedliche Strecken}
        \begin{leftbar}[0.9\linewidth]
            Unterschiedliches Straßen Assets in Tunneln, Kuppeln und Räumen.
        \end{leftbar}

    \item \textit{Neue Biome}
        \begin{leftbar}[0.9\linewidth]
            Es wurden 4 fertige Biome definiert, die sich in den Punkten Untergrund, Farben und platzierten Objekten
            unterscheiden.
        \end{leftbar}

    \item \textit{Kuppeldach wurde durch Asset ersetzt}
        \begin{leftbar}[0.9\linewidth]
            Das Dach einer Kuppel besteht nun aus einem, in Blender erstelltem, Asset.
        \end{leftbar}

    \item \textit{Definition von 3 Kartentypen}
        \begin{leftbar}[0.9\linewidth]
            Es wurden ein kleines, ein mittleres und ein großes Biom definiert, die sich in der Anzahl der Kuppeln, 
            Tunneln und Räumen unterscheiden. Ebenso sind di möglichen Größen eines solchen \textit{ISceneItems} unterschiedlich.
        \end{leftbar}
\end{itemize} 