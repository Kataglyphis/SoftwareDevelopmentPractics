\subsection{InGame - Grafik}

\subsubsection{Neue Features}
\begin{itemize}
    \item \textit{Testfeature}
        \begin{leftbar}[0.9\linewidth]
            Beschreibung des Features
        \end{leftbar}
\end{itemize}

\subsubsection{Änderungen existierender Komponenten}
\begin{itemize}
    \item \textit{Testfeature}
        \begin{leftbar}[0.9\linewidth]
            Beschreibung der Änderung und evtl Begründung
        \end{leftbar}
\end{itemize}

\subsubsection{Nicht implementiert}
\begin{itemize}
    \item \textit{TestNotExistingfeature}
        \begin{leftbar}[0.9\linewidth]
            Alle Methoden die beim Partikeleffekt abhängig von tpf sind wurden nicht
            mit einen JUnit test getestet. Da Ergebnisse abhängig von Framerate macht
            ein klassischer JUnit Test keinen Sinn. Diese wurden allerdings ausführlich
            grafisch auf Funktionalität getestet. 
        \end{leftbar}
\end{itemize}